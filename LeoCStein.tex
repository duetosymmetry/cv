\let\nofiles\relax % This is because res says to not emit aux files,
                   % but lastpage needs aux files.
\documentclass[margin,line]{res}
\usepackage[colorlinks=true]{hyperref}
\usepackage[utf8]{inputenc}
\usepackage[T1]{fontenc}
\usepackage{microtype}
\usepackage{fourier-orns}
\usepackage{amsmath,amssymb}
\usepackage{lastpage}
\usepackage{fancyhdr}
\usepackage{etaremune}
\usepackage[normalem]{ulem}
\usepackage[style=iso]{datetime2}

\oddsidemargin -.5in
\evensidemargin -.5in
\voffset -25pt
%\topmargin -.2in
\headsep 25pt
\textwidth=6.0in
\textheight=8.9in
\itemsep=0in
\parsep=0in

% Headings
\pagestyle{fancy}
\lhead{Leo C.~Stein --- Curriculum Vitae (as of \today)}
\chead{}
\rhead{\thepage\ of \pageref*{LastPage}}
\lfoot{}
\cfoot{}
\rfoot{}
\renewcommand{\headrulewidth}{0.4pt}
%\renewcommand{\footrulewidth}{0.4pt}

% Give hyperref some metadata
\hypersetup{pdftitle={Leo C. Stein — Curriculum Vitae (CV)},
  pdfauthor={Leo C. Stein},
  pdfsubject={Leo C. Stein's academic curriculum vitae (CV)},
  pdfkeywords={physics, gravity, general relativity, black holes,
    gravitational waves, numerical relativity, beyond-GR, asymptotia,
    differential geometry, dynamical systems}}

% I want small caps for the section style
\def\sectionfont{\sc}

% Generate a PDF TOC
\let\oldsection\section
\def\section#1{\oldsection{#1}%
\phantomsection%
\addcontentsline{toc}{section}{#1}%
}

% The above seems to screw up the spacing for sections that start with lists...
\newcommand{\secstartswithlist}{\leavevmode \vspace{-\baselineskip}}

\begin{document}

\newcommand{\myname}{Leo C.~Stein}
\newlength{\mynamewidth}
\settowidth{\mynamewidth}{\namefont\myname}

\name{\hspace*{0.5\textwidth}\hspace{-0.5\mynamewidth} \myname \vspace*{.1in}}
% On the first page, have no header.
\thispagestyle{empty}

\begin{resume}

\section{Contact Information}
%\vspace{.05in}
205 Lewis Hall      \hfill \href{mailto:lcstein@olemiss.edu}{lcstein@olemiss.edu}\\
University of Mississippi   \hfill \href{https://duetosymmetry.com/}{duetosymmetry.com}\\
University, MS 38677-1848 USA \hfill \href{tel:1-662-915-1941}{1-662-915-1941}

%%%%%%%%%%%%%%%%%%%%%%%%%%%%%%%%%%%%%%%%%%%%

% Local Variables:
% mode: latex
% TeX-master: "LeoCStein.tex"
% End:


% \section{Citizenship}
% United States

\section{Education}
{\bf Ph.D., Physics,} Massachusetts Institute of Technology, Cambridge, MA, USA \hfill {\bf May 2012}\\
\vspace*{-.1in}
\begin{itemize}
\item[ ] Dissertation Advisor: Prof.\ Scott Hughes
\item[ ] Dissertation Title: {\it Probes of strong-field gravity}
%\item[ ] G.P.A.: 4.6/5.0
\end{itemize}

{\bf B.S., Physics,} California Institute of
Technology, Pasadena, CA, USA \hfill {\bf June 2006}\\
\vspace*{-.1in}
\begin{itemize}
\item[ ] Degree conferred with honor.
\item[ ] Senior Thesis Advisors: Dr.\ Patrick Sutton and Prof.\ Alan Weinstein
%\item[ ] G.P.A.: ?.??/4.00
\end{itemize}

\section{Employment}
{\bf Associate Professor,} University of Mississippi, Oxford, MS USA
\hfill {\bf July 2024--Present}\\
\\
{\bf Assistant Professor,} University of Mississippi, Oxford, MS USA
\hfill {\bf August 2018--June 2024}\\
\\
{\bf Senior Postdoctoral Researcher,} Caltech, Pasadena, CA USA
\hfill {\bf September 2015--August 2018}\\
\\
{\bf NASA Einstein Fellow,} Cornell, Ithaca NY, USA
\hfill {\bf September 2012--August 2015}\\
\\
{\bf Research and Teaching Assistant,} MIT, Cambridge MA, USA \hfill {\bf September 2006--May 2012}\\
\\
{\bf Teaching Assistant,} Caltech, Pasadena, CA, USA \hfill {\bf  Fall 2004, Spring 2005}\\
\\
{\bf Summer Research Fellow,} Caltech, Pasadena, CA, USA \hfill {\bf  June--September 2003/2005}\\

% Added to improve page breaks
\vspace{-1em}

\section{Research Interests}
General relativity (GR), gravitation, and astrophysical phenomena which can
elucidate gravity.
One major theme is pushing numerical and analytical gravitational-wave
(GW) predictions to the precision frontier in advance of
next-generation observatories.
A second major theme is using GWs to test GR against beyond-GR
models, in both theory-independent and theory-dependent models.
This involves numerical relativity and renormalization methods
applied to specific effective field models for beyond-GR theories.

\section{Honors and Awards}
{\bf Kavli Fellow,}
National Academy of Sciences,
\hfill {\bf 2025}\\
\\
{\bf Sloan Research Fellowship,}
Alfred P.\ Sloan Foundation,
\hfill {\bf 2023--2025}\\
\\
{\bf CAREER Award,} NSF \hfill {\bf 2021--2026}\\
\\
{\bf Einstein Postdoctoral Fellow,} NASA \hfill {\bf 2012--2015}\\
\\
{\bf Henry Kendall Teaching Award,} Massachusetts Institute of Technology \hfill {\bf 2011}\\
\\
{\bf Upperclass Merit Scholarship,} California Institute of Technology \hfill {\bf 2005--2006}\\

% Added to improve page breaks
%\vspace{-1em}

\section{Teaching Experience}
{\bf Professor}, University of Mississippi
\vspace*{.05in}
\begin{itemize}
\item[ ] Phys.~213, General physics I \hfill {\bf Spring 2021}
\item[ ] Phys.~401, Electromagnetism I \hfill {\bf Falls 2019--2022}
\item[ ] Phys.~402, Electromagnetism II \hfill {\bf Springs 2019--2021}
\item[ ] Phys.~436, Intro to cosmology \hfill {\bf Fall 2023}
\item[ ] Phys.~463/4, Senior research project \hfill {\bf Fall 2020,
    Spring 2021, Fall 2023}
\item[ ] Phys.~503/630, Graduate reading course \hfill {\bf Spring 2019, Falls 2020--2021}
\item[ ] Phys.~709, Graduate classical dynamics I \hfill {\bf Fall 2018}
\item[ ] Phys.~721, Graduate electrodynamics I \hfill {\bf Springs 2022--2025}
\item[ ] Phys.~722, Graduate electrodynamics II \hfill {\bf Falls 2022--2024}
\item[ ] Phys.~735, General relativity \hfill {\bf Fall 2024}
\item[ ] Phys.~750, General relativity II \hfill {\bf Spring 2020}
\end{itemize}
%\newpage{}
{\bf Guest Lecturer}, California Institute of Technology
\vspace*{.05in}
\begin{itemize}
\item[ ] Ph236, General relativity \hfill {\bf Fall 2017}
\item[ ] Ph237, Gravitational Waves \hfill {\bf Spring 2016}
\end{itemize}
{\bf Guest Lecturer}, Massachusetts Institute of Technology
\vspace*{.05in}  
\begin{itemize}
\item[ ] 8.901, Graduate Astrophysics I \hfill {\bf Spring 2011}
\end{itemize}
{\bf Teaching Assistant}, Massachusetts Institute of Technology
\vspace*{.05in}
\begin{itemize}
\item[ ] 8.942, Cosmology \hfill {\bf Fall 2011}
\item[ ] 8.901, Graduate Astrophysics I \hfill {\bf Spring 2011}
\item[ ] 8.286, The Early Universe \hfill {\bf Fall 2009}
\end{itemize}
{\bf Teaching Assistant}, California Institute of Technology
\vspace*{.05in}
\begin{itemize}
\item[ ] Ph 7, Nuclear and Quantum Physics Lab\hfill {\bf Spring 2005}
\item[ ] Ph 5, Analog Electronics for Physicists \hfill {\bf Fall 2004}
\end{itemize}

\section{Mentoring/ Supervision}
{\bf Postdoctoral researchers}
\begin{itemize}
\item[] Károly Csukás
\hfill {\bf Fall 2021--Summer 2024}
\item[] José Tomás Gálvez Ghersi
\hfill {\bf Fall 2019--Spring 2021}
\begin{itemize}
\vspace{-.05in}
\item[] Now faculty at Universidad de Ingeniería y Tecnología, Peru
\vspace{-.05in}
\end{itemize}
\end{itemize}
{\bf Graduate students}
\begin{itemize}
\item[] David Bronicki, University of Mississippi
\hfill {\bf Fall 2019--Summer 2023}
\item[] Subhayu Bagchi, University of Mississippi
\hfill {\bf Fall 2019--present}
\item[] Aniket Khairnar, University of Mississippi
\hfill {\bf Fall 2019--present}
\item[] Akshay Khadse, University of Mississippi
\hfill {\bf Fall 2018--Summer 2024}
\item[] Lorena Magaña Zertuche, University of Mississippi
\hfill {\bf Fall 2018--Summer 2024}
\begin{itemize}
\vspace{-.05in}
\item[] Now a postdoc at NBI, Copenhagen, Denmark
\end{itemize}
\item[] Joe Rivest, University of Mississippi
\hfill {\bf Fall 2018--Summer 2024}
\item[] Sashwat Tanay, University of Mississippi
\hfill {\bf Fall 2018--Summer 2022}
\begin{itemize}
\vspace{-.05in}
\item[] Now a postdoc at LUTH, Meudon, France
\end{itemize}
\item[] Maria (Masha) Okounkova, Caltech
\hfill {\bf Fall 2015--Summer 2019}
\begin{itemize}
\vspace{-.05in}
\item[] Now faculty at Pasadena City College
\end{itemize}
\item[] Baoyi Chen, Caltech
\hfill {\bf Fall 2016--Summer 2018}
\end{itemize}

%
%\newpage{}

{\bf Undergraduate students}
\begin{itemize}
\item[] Wayne Zhao, Harvard
\hfill {\bf Summer 2016}
\begin{itemize}
\vspace{-.05in}
\item[] Now a graduate student at Princeton
\end{itemize}
\end{itemize}

\section{Professional Activities, Outreach, and Service}
{\bf LISA Consortium, Full member}\hfill{\bf 2020--Present}
\begin{itemize}
\item[] UMiss LISA Group leader\hfill{\bf 2020--Present}
\end{itemize}
{\bf Simulating eXtreme Spacetimes collaboration}\hfill{\bf 2015--Present}
\begin{itemize}
\item[] Executive committee member\hfill{\bf 2018--Present}
\end{itemize}
{\bf American Physical Society, member}\hfill{\bf 2010--Present}
\begin{itemize}
\item[] Division of Gravitational Physics
  \begin{itemize}
  \item[] Secretary/Treasurer\hfill{\bf 2023--2026}
  \item[] Executive Committee Member-at-Large\hfill{\bf 2016--2019}
  \end{itemize}
\item[] Division of Astrophysics
\end{itemize}
{\bf Conference organizer}
\vspace*{.05in}
\begin{itemize}
\item[]
  \href{https://indico.global/event/14228/}
  {Nonlinear Black Hole Perturbation Theory},
U of Nottingham
\hfill {\bf September 2025}\\
\hspace*{1em} Three day workshop, $\sim 70$ participants
\item[]
  \href{https://www.phy.olemiss.edu/gcgm11/}
  {11\textsuperscript{th} Gulf Coast Gravity Meeting (GCGM)},
UMiss
\hfill {\bf April 2025}\\
\hspace*{1em} Two day conference, $\sim 50$ participants
\item[]
  \href{https://icerm.brown.edu/topical_workshops/tw-24-ses/}
  {Simulating Extreme Spacetimes with SpEC and SpECTRE},
  ICERM \hfill {\bf August 2024} \\
\hspace*{1em} Week-long international workshop, $\sim$85 participants
\item[]
  \href{https://www.benasque.org/2024relativity/}
  {New frontiers in strong gravity},
  Benasque, Spain \hfill {\bf July 2024} \\
\hspace*{1em} Two week international conference, $\sim$70 participants
\item[]
  \href{https://pcts.princeton.edu/events/2023/nonlinear-aspects-general-relativity}
  {Nonlinear Aspects of General Relativity},
  Princeton PCTS \hfill {\bf October 2023}\\
\hspace*{1em} Four day workshop, $\sim$100 participants
\item[]
  \href{https://icerm.brown.edu/events/re-22-f20w1/}
  {Numerical Relativity Community Summer School},
  ICERM \hfill {\bf August 2022} \\
\hspace*{1em} Week-long international summer school, 150 participants
\item[]
  \href{https://www.benasque.org/2022relativity/}
  {New frontiers in strong gravity},
  Benasque, Spain \hfill {\bf July 2022} \\
\hspace*{1em} Two week international conference, 100 participants
\item[]
  \href{http://www.benasque.org/2018relativity/}
  {Numerical Relativity beyond General Relativity},
  Benasque, Spain \hfill {\bf June 2018} \\
\hspace*{1em} Week-long international workshop, 59 participants
\item[]
34\textsuperscript{th} Pacific Coast Gravity Meeting (PCGM),
Caltech
\hfill {\bf March 2018}\\
\hspace*{1em} Two day conference, $\sim 125$ participants
\item[]
  \href{http://www.tapir.caltech.edu/~unifying-gr-tests/}
  {Unifying Tests of General Relativity},
  Caltech \hfill {\bf July 2016} \\
\hspace*{1em} Three day workshop, 52 participants
\end{itemize}
{\bf Seminar organizer}
\vspace*{.05in}
\begin{itemize}
\item[] TAPIR seminar, Caltech\hfill
  {\bf Fall 2015--Spring 2018}
\item[] General Relativity Informal Tea-Time Series (GRITTS), MIT\hfill
  {\bf Fall 2011--Spring 2012}
\item[] MKI Journal Club, MIT\hfill {\bf Fall 2007--Spring 2010}
\end{itemize}
{\bf Conference session chair; Judge for best student speaker award}
\vspace*{.05in}
\begin{itemize}
\item[]
April APS meeting, NY, NY
\hfill {\bf April 2022}
\item[]
Midwest relativity meeting, Grand Rapids, MI
\hfill {\bf October 2019}
\item[]
April APS meeting, Columbus, OH
\hfill {\bf April 2018}
\item[]
34\textsuperscript{th} Pacific Coast Gravity Meeting (PCGM),
Caltech
\hfill {\bf March 2018}
\item[]
33\textsuperscript{rd} Pacific Coast Gravity Meeting (PCGM),
UCSB
\hfill {\bf March 2017}
\item[]
``April'' APS meeting, Washington D.C.
\hfill {\bf January 2017}
\item[]
32\textsuperscript{nd} Pacific Coast Gravity Meeting (PCGM),
CSU Fullerton
\hfill {\bf April 2016}
\item[]
Theoretical Astrophysics in Southern California (TASC),
CSU Fullerton
\hfill {\bf November 2015}
\end{itemize}

{\bf Journal referee}
\vspace*{.05in}\\
\hspace*{1em}
American Journal of Physics,
Classical and Quantum Gravity,
Journal of Cosmology and Astroparticle Physics,
Journal of Open Source Software,
General Relativity and Gravitation,
Monthly Notices of the Royal Astronomical Society,
Physics Letters~B,
Physical Review D,
Physical Review Letters,
Physical Review X,
Reviews of Modern Physics,
The Astrophysical Journal Letters,
The~Physics~Teacher

{\bf Agency work}
\vspace*{.05in}\\
\hspace*{1em}
Reviewer for NSF, NASA

{\bf Outreach}
\vspace*{.05in}
\begin{itemize}

\item[] Oxford Science Café
  \hfill {\bf April 2019} \\
  \hspace*{1em} Lecture: ``The truth about black holes''

\item[] Guest on the {\it Starts With a Bang} podcast
  \hfill {\bf March 25, 2019} \\
  \hspace*{1em}
\href{https://soundcloud.com/ethan-siegel-172073460/starts-with-a-bang-42-black-holes-and-gravitation}{Episode 42: Black holes and gravitation}


\item[] Invited speaker for Latin American Webinar on Physics
  \hfill {\bf March 13, 2019} \\
  \hspace*{1em} \href{https://www.youtube.com/watch?v=7HO07-QtvMI}
  {Webinar 75: ``Testing Einstein with numerical relativity''}

\item[] Caltech astronomy public lecture series speaker
  \hfill {\bf March 2018} \\
  \hspace*{1em} Lecture: ``The truth about black holes''

\item[] Astronomy on Tap public lecture series speaker and volunteer
  \hfill {\bf 2016--2018} \\
\hspace*{1em} Close to a monthly basis

\item[] Caltech astronomy public lecture series panelist and emcee
        \hfill {\bf 2016--2018} \\
\hspace*{1em} Approximately every three months

\item[] Invited guest lecture on black holes and gravitational waves \hfill {\bf November 2017} \\
\hspace*{1em} {\it Science of Space and Time}, Hampshire College

\item[] Invited video Q\&A session, public high school physics class \hfill {\bf June 2017} \\
\hspace*{1em} {\it The Nova Project} school, Seattle

\item[] Guest on {\it The Titanium Physicists Podcast} \\
\hspace*{1em}
\href{http://titaniumphysicists.brachiolopemedia.com/2019/08/21/episode-80-picturing-the-bach-hole-with-adal-rifai/}{Episode
  80: Picturing the Bach Hole}
\hfill
{\bf August 21, 2019} \\
\hspace*{1em} \href{http://titaniumphysicists.brachiolopemedia.com/2016/04/25/episode-64-e-and-n-the-edges-of-einstein/}{Episode 64: The edges of Einstein}
\hfill
{\bf April 25, 2016} \\
\hspace*{1em} \href{http://titaniumphysicists.brachiolopemedia.com/2016/02/01/episode-62-black-bells-with-brent-knopf-and-matt-sheehy/}{Episode 62: Black Bells}
\hfill
{\bf February 1, 2016} \\

\item[] Quora \href{https://www.quora.com/session/Leo-C-Stein/1}{Q\&A Session} on gravitational waves and first detection
  \hfill {\bf February 17, 2016} \\
\hspace*{1em} 83.9k+ views, 20.8k+ followers

\item[] Invited guest host, public screening of {\it COSMOS} with Q\&A,
  \hfill {\bf March/June 2014} \\
\hspace*{1em} Science Cabaret/Cornell
\item[] Invited public talk at {\it Frontiers of Cornell Astronomy}, \hfill {\bf November 2013} \\
\hspace*{1em} Cornell Friends of Astronomy
\item[] Invited video chat, {\it Topics in Physics} course, \hfill {\bf July 2013} \\
\hspace*{1em} Stanford Education Program for Gifted Youth
\end{itemize}

\section{Computer Skills}
%{\bf Languages---}%
Expert in {\sc Mathematica}, C/C++, Python, Bash.
Proficient in Javascript.
Experience in Haskell, Java, Julia.
Expert at *nix and HPC.
Markup languages: \LaTeX, HTML, CSS, Markdown.

% {\bf Operating systems---}%
% Mac OS, Linux/*nix.

{\bf Software---}%
Most contributions can be found at \url{https://github.com/duetosymmetry}.
Member of the {\it Simulating eXtreme Spacetimes} (SXS) collaboration,
contributor to the Spectral Einstein Code (SpEC).
Member of the {\it Black Hole Perturbation Toolkit}. Author of
{\tt qnm} python package (\url{https://github.com/duetosymmetry/qnm}).
Core collaborator on {\sc xAct} (\url{http://xact.es}) abstract
tensor calculus package for {\sc  Mathematica}. Coauthor
of {\sc xTerior} package for exterior differential geometry under
{\sc xAct}. Co-maintainer of community contributions at
\url{http://contrib.xact.es}. Developed
\href{https://chrome.google.com/webstore/detail/arxiv-keys/fkjjdlbhliopfhgddlpoggpmpgjfaojd}{arXiv-keys}
browser extension/add-on for Chrome/Firefox.
Author of \href{https://ctan.org/pkg/orcidlink}{\tt orcidlink} and coauthor of \href{https://ctan.org/pkg/gridpapers}{\tt gridpapers} packages for \LaTeX.

% Twiddle this for spacing
% \newpage

\ifx\nopubs\undefined
\newcommand{\arxiv}[1]{[\href{http://arxiv.org/abs/#1}{arXiv:#1}]}
% Citation counts last updated 2023-08-06
\def\zero{0}
\def\one{1}
\newcommand{\citeCount}[1]{%
  \def\val{#1}
  \ifx\val\zero%
  \else%
    \ifx\val\one%
    (1~citation)%
    \else%
    (#1~citations)%
    \fi%
  \fi}

% Comment out to show, uncomment to hide
\renewcommand{\citeCount}[1]{}

\newcounter{numPubs}
\newcounter{pubCounter}

\setcounter{numPubs}{63}
\setcounter{pubCounter}{\value{numPubs}}

% \section{\sc Publications in Progress}
% \begin{etaremune}[start=\value{pubCounter}]
% \item
%   McNees,~R.
%   {\bf Stein,~L.~C.},
%   (2019)
%   {\it Cosmological perturbations in dynamical Chern-Simons}.
%   \setcounter{pubCounter}{\value{enumi}}
% \end{etaremune}

%%%%%%%%%%
%%%%%%%%%%
%%% As of 2023-08-06
%%%%%%%%%%
%%%%%%%%%%
\newif\ifshowpubsummary
%%% Comment out the next line to omit the publication summary
\showpubsummarytrue
%%%
\ifshowpubsummary
\section{\sc Publication Summary}
{\bf h-index ---}%
As of 2024-10-13: 62 (according to Google Scholar), or 55 (according
to INSPIRE).  Both include collaboration papers.

{\bf Top five cited ---}%
Excluding LIGO/Virgo collaboration papers.
%Citation counts from Google Scholar.
\begin{enumerate}
\item
  Berti, E., (5 authors), {\bf Stein,~L.~C.}, (46 more authors)
  (2015)
  {\it Testing General Relativity with Present and Future
    Astrophysical Observations},
  \href{http://dx.doi.org/10.1088/0264-9381/32/24/243001}{Class. Quantum Grav. {\bf 32} 243001}
  \arxiv{1501.07274}.
  \citeCount{1462}
\item
  Barack,~L., {\it et al.}
  (2019)
  {\it Black holes, gravitational waves and fundamental physics: a roadmap},
  \href{https://doi.org/10.1088/1361-6382/ab0587}{Class.~Quantum Grav.~{\bf 36} 143001}
  \arxiv{1806.05195}.
  \citeCount{763}
\item
  Boyle,~M., {\it et al.} ({\bf LCS} is corresponding author)
  (2019)
  {\it The SXS Collaboration catalog of binary black hole simulations},
  \href{https://doi.org/10.1088/1361-6382/ab34e2}{Class.~Quantum Grav.~{\bf 36} 195006}
  \arxiv{1904.04831}.
  \citeCount{400}
\item
  Varma,~V, {\it et al.}
  (2019)
  {\it Surrogate models for precessing binary black hole simulations with
  unequal masses},
  \href{https://doi.org/10.1103/PhysRevResearch.1.033015}{Phys.~Rev.~Research~{\bf 1},~033015}
  \arxiv{1905.09300}.
  \citeCount{374}
\item
  Yunes,~N., {\bf Stein,~L.~C.}
  (2011),
  {\it Nonspinning black holes in alternative theories of gravity},
  \href{http://dx.doi.org/10.1103/PhysRevD.83.104002}{Phys.~Rev.~D~{\bf 83}~104002}
  \arxiv{1101.2921}.
  \citeCount{330}
\end{enumerate}
\else% don't show pub summary
\fi

\renewcommand{\citeCount}[1]{}

\section{\sc Submitted Publications}
%\addtocounter{pubCounter}{-1}
\begin{etaremune}[start=\value{pubCounter}]
\item
  Khairnar,~A.,
  {\bf Stein,~L.~C.},
  Boyle,~M.,
  (2024)
  \\
  {\it Approximate helical symmetry in compact binaries}.
  \citeCount{0}
\item
  Magaña~Zertuche,~L.,
  {\bf Stein,~L.~C.},
  {\it et al.},
  (2024)
  {\it High-Precision Ringdown Surrogate Model for Non-Precessing Binary Black Holes},
  \arxiv{2408.05300}.
  \citeCount{0}
\item
  Zhu,~H.,
  (9 authors),
  {\bf Stein,~L.~C.},
  (2024)
  {\it Imprints of Changing Mass and Spin on Black Hole Ringdown},
  \arxiv{2404.12424}.
  \citeCount{0}
\item
  Sun,~D.,
  Boyle,~M.,
  Mitman,~K.,
  Scheel,~M.~A.,
  {\bf Stein,~L.~C.},
  Teukolsky,~S.~A.,
  Varma,~V.,
  (2024)
  {\it Optimizing post-Newtonian parameters and fixing the BMS frame for numerical-relativity waveform hybridizations},
  \arxiv{2403.10278}.
  \citeCount{0}
  \setcounter{pubCounter}{\value{enumi}}
\end{etaremune}

% \section{\sc Accepted Publications}
% \addtocounter{pubCounter}{-1}
% \begin{etaremune}[start=\value{pubCounter}]
% \item
%   Mitman,~K.,
%   Boyle,~M.,
%   {\bf Stein,~L.~C.},
%   {\it et al.},
%   (2024)
%   {\it A Review of Gravitational Memory and BMS Frame Fixing in Numerical Relativity},
%   \arxiv{2405.08868}.
%   Accepted to CQG.
%   \citeCount{0}
%   \setcounter{pubCounter}{\value{enumi}}
% \end{etaremune}

\section{\sc Collaboration Publications}
From 2008--2012, I was coauthor on 34 refereed LIGO and/or LIGO/Virgo
collaboration publications. I only list short author-list publications below.

\section{\sc Refereed Publications}
\addtocounter{pubCounter}{-1}
\begin{etaremune}[start=\value{pubCounter}]
\item
  Mitman,~K.,
  Boyle,~M.,
  {\bf Stein,~L.~C.},
  {\it et al.},
  (2024)
  {\it A Review of Gravitational Memory and BMS Frame Fixing in Numerical Relativity},
  \href{https://doi.org/10.1088/1361-6382/ad83c2}%
  {Class.~Quantum Grav.~{\bf 41} 223001},
  \arxiv{2405.08868}.
  \citeCount{0}
\item
  {\bf Stein,~L.~C.},
  (2024)
  {\it Can a radiation gauge be horizon-locking?},
  \href{https://doi.org/10.1088/1361-6382/ad563b}%
  {Class.~Quantum Grav.~{\bf 41} 157001}
  \arxiv{2404.10113}.
  \citeCount{0}
\item
  Samanta,~R.,
  Tanay,~S.,
  {\bf Stein,~L.~C.},
  (2023)
  {\it Closed-form solutions of spinning, eccentric binary black holes at 1.5 post-Newtonian order},
  \href{https://doi.org/10.1103/PhysRevD.108.124039}%
  {Phys.~Rev.~D~{\bf 108},~124039}
  \arxiv{2210.01605}.
  \citeCount{2}
\item
  Bronicki,~D.,
  Cárdenas-Avendaño,~A.,
  {\bf Stein,~L.~C.},
  (2023)
  {\it Tidally-induced nonlinear resonances in EMRIs with an analogue model},
  \href{https://doi.org/10.1088/1361-6382/acfcfe}%
  {Class.~Quantum Grav.~{\bf 40} 215015}
  \arxiv{2203.08841}.
  \citeCount{8}
\item
  Yoo,~J., {\it et al.},
  (2023)
  {\it Numerical relativity surrogate model with memory effects and post-Newtonian hybridization},
  \href{https://doi.org/10.1103/PhysRevD.108.064027}%
  {Phys.~Rev.~D~{\bf 108},~064027}
  \arxiv{2306.03148}.
  \citeCount{1}
\item
  Ma,~S.,
  Varma,~V.,
  {\bf Stein,~L.~C.},
  {\it et al.}
  (2023)
  {\it Numerical simulations of black hole--neutron star mergers in scalar-tensor gravity},
  \href{https://doi.org/10.1103/PhysRevD.107.124051}%
  {Phys.~Rev.~D~{\bf 107},~124051}
  \arxiv{2304.11836}.
  \citeCount{1}
\item
  Tanay,~S.,
  {\bf Stein,~L.~C.},
  Cho,~G.,
  (2023)
  {\it Action-angle variables of a binary black-hole with arbitrary eccentricity, spins, and masses at 1.5 post-Newtonian order},
  \href{https://doi.org/10.1103/PhysRevD.107.103040}%
  {Phys.~Rev.~D~{\bf 107},~103040}
  \arxiv{2110.15351}.
  \citeCount{7}
\item
  Grant,~A.~M.,
  Saffer,~A.,
  {\bf Stein,~L.~C.},
  Tahura,~A.,
  (2023)
  {\it Gravitational-wave energy and other fluxes in ghost-free bigravity},
  \href{https://doi.org/10.1103/PhysRevD.107.044041}%
  {Phys.~Rev.~D~{\bf 107},~044041}
  \arxiv{2208.02123}.
  \citeCount{1}
\item
  Mitman,~K.,
  Lagos,~M.,
  {\bf Stein,~L.~C.},
  {\it et al.}
  (2023)
  {\it Nonlinearities in black hole ringdowns},
  \href{https://doi.org/10.1103/PhysRevLett.130.081402}{Phys.~Rev.~Lett.~{\bf 130},~081402}
  \arxiv{2208.07380}.
  \raisebox{.15em}{\aldine} Editors' Suggestion,
  \href{https://physics.aps.org/articles/v16/29}{Featured in Physics}.
  \citeCount{46}
\item
  Clark,~W.~A.,
  Gomes,~M.~W.,
  Rodriguez-Gonzalez,~A.,
  {\bf Stein,~L.~C.},
  Strogatz,~S.~H.,
  (2023)
  {\it Surprises in a classic boundary-layer problem},
  \href{https://doi.org/10.1137/21M1436087}%
  {SIAM Review 2023 65:1, 291-315}
  \arxiv{2107.11624}.
  \citeCount{1}
\item
  Mitman,~K.,
  {\bf Stein,~L.~C.},
  Boyle,~M., {\it et al.}
  (2022)
  {\it Fixing the BMS Frame of Numerical Relativity Waveforms with BMS Charges},
  \href{https://doi.org/10.1103/PhysRevD.106.084029}%
  {Phys.~Rev.~D~{\bf 106},~084029}
  \arxiv{2208.04356}.
  \citeCount{7}
\item
  Okounkova,~M,
  Farr,~W.~M.,
  Isi,~M.,
  {\bf Stein,~L.~C.},
  (2022)
  {\it Constraining gravitational wave amplitude birefringence and Chern-Simons gravity with GWTC-2},
  \href{https://doi.org/10.1103/PhysRevD.106.044067}%
  {Phys.~Rev.~D~{\bf 106},~044067}
  \arxiv{2101.11153}.
  \citeCount{39}
\item
  Magaña~Zertuche,~L.,
  Mitman,~K.,
  Khera,~N.,
  {\bf Stein,~L.~C.},
  et~al.,
  (2022)
  {\it High Precision Ringdown Modeling: Multimode Fits and BMS Frames},
  \href{https://doi.org/10.1103/PhysRevD.105.104015}%
  {Phys.~Rev.~D~{\bf 105},~104015}
  \arxiv{2110.15922}.
  \citeCount{27}
\item
  Gálvez Ghersi,~J.~T.,
  {\bf Stein,~L.~C.},
  (2021)
  {\it Numerical renormalization group-based approach to secular perturbation theory},
  \href{https://doi.org/10.1103/PhysRevE.104.034219}%
  {Phys.~Rev.~E~{\bf 104},~034219}
  \arxiv{2106.08410}.
  \citeCount{11}
\item
  Mitman,~K.,
  Khera,~N.,
  Iozzo,~D.~A.~B.,
  {\bf Stein,~L.~C.},
  et~al.,
  (2021)
  {\it Fixing the BMS frame of numerical relativity waveforms},
  \href{https://doi.org/10.1103/PhysRevD.104.024051}%
  {Phys.~Rev.~D~{\bf 104},~024051}
  \arxiv{2105.02300}.
  \citeCount{17}
\item
  Iozzo,~D.~A.~B.,
  Khera,~N.,
  {\bf Stein,~L.~C.},
  et~al.,
  (2021)
  {\it Comparing Remnant Properties from Horizon Data and Asymptotic Data in Numerical Relativity},
  \href{https://doi.org/10.1103/PhysRevD.103.124029}%
  {Phys.~Rev.~D~{\bf 103},~124029}
  \arxiv{2104.07052}.
  \citeCount{15}
\item
  Tahura,~S.,
  Nichols,~D.~A.,
  Saffer,~A.,
  {\bf Stein,~L.~C.},
  Yagi,~K.
  (2020)
  {\it Brans-Dicke theory in Bondi-Sachs form: Asymptotically flat solutions, asymptotic symmetries and gravitational-wave memory effects},
  \href{https://doi.org/10.1103/PhysRevD.103.104026}%
  {Phys.~Rev.~D~{\bf 103},~104026}
  \arxiv{2007.13799}.
  \citeCount{27}
\item
  Tanay,~S.,
  {\bf Stein,~L.~C.},
  Gálvez~Ghersi,~J.~T.,
  (2020)
  {\it Integrability of eccentric, spinning black hole binaries up to second post-Newtonian order},
  \href{https://doi.org/10.1103/PhysRevD.103.064066}%
  {Phys.~Rev.~D~{\bf 103},~064066}
  \arxiv{2012.06586}.
  \citeCount{13}
\item
  Gálvez~Ghersi,~J.~T.,
  {\bf Stein,~L.~C.},
  (2020)
  {\it A fixed point for black hole distributions},
  \href{https://doi.org/10.1088/1361-6382/abcfd2}
  {Class.~Quantum Grav.~{\bf 38} 045012}
  \arxiv{2007.11578}.
  \citeCount{6}
\item
  Okounkova,~M.,
  {\bf Stein,~L.~C.},
  Moxon,~J.,
  Scheel,~M.~A.,
  Teukolsky,~S.~A.,
  (2020)
  {\it Numerical relativity simulation of GW150914 beyond general relativity},
  \href{https://doi.org/10.1103/PhysRevD.101.104016}{Phys.~Rev.~D~{\bf 101},~104016}
  \arxiv{1911.02588}.
  \citeCount{73}
\item
  {\bf Stein,~L.~C.},
  Warburton,~N.,
  (2020)
  {\it Location of the last stable orbit in Kerr spacetime},
  \href{https://doi.org/10.1103/PhysRevD.101.064007}{Phys.~Rev.~D~{\bf 101},~064007}
  \arxiv{1912.07609}.
  \citeCount{34}
\item
  Okounkova,~M.,
  {\bf Stein,~L.~C.},
  Scheel,~M.~A.,
  Teukolsky,~S.~A.,
  (2019)
  {\it Numerical binary black hole collisions in dynamical Chern-Simons gravity},
  \href{https://doi.org/10.1103/PhysRevD.100.104026}{Phys.~Rev.~D~{\bf 100},~104026}
  \arxiv{1906.08789}.
  \citeCount{82}
\item
  Varma,~V, {\it et al.}
  (2019)
  {\it Surrogate models for precessing binary black hole simulations with
  unequal masses},
  \href{https://doi.org/10.1103/PhysRevResearch.1.033015}{Phys.~Rev.~Research~{\bf 1},~033015}
  \arxiv{1905.09300}.
  \citeCount{205}
\item
  {\bf Stein,~L.~C.},
  (2019)
  \hspace{0.1em}
  {\it {\tt qnm:} A Python package for calculating Kerr quasinormal modes, separation constants, and spherical-spheroidal mixing coefficients},
  \href{https://doi.org/10.21105/joss.01683}{J.~Open~Source~Softw., 4(42), 1683}
  \arxiv{1908.10377}.
  \citeCount{43}
\item
  Boyle,~M., {\it et al.} ({\bf LCS} is corresponding author)
  (2019)
  {\it The SXS Collaboration catalog of binary black hole simulations},
  \href{https://doi.org/10.1088/1361-6382/ab34e2}{Class.~Quantum Grav.~{\bf 36} 195006}
  \arxiv{1904.04831}.
  \citeCount{245}
\item
  Barack,~L., {\it et al.}
  (2019)
  {\it Black holes, gravitational waves and fundamental physics: a roadmap},
  \href{https://doi.org/10.1088/1361-6382/ab0587}{Class.~Quantum Grav.~{\bf 36} 143001}
  \arxiv{1806.05195}.
  \citeCount{592}
\item
  Varma,~V., {\bf Stein,~L.~C.}, Gerosa,~D.,
  (2019)
  {\it The binary black hole explorer: on-the-fly visualizations of precessing binary black holes},
  \href{https://doi.org/10.1088/1361-6382/ab0ee9}{Class.~Quantum Grav.~{\bf 36} 095007}
  \arxiv{1811.06552},
  [\href{https://vijayvarma392.github.io/binaryBHexp/}{project~website}].
  \citeCount{2}
\item
  Varma,~V., Gerosa,~D., {\bf Stein,~L.~C.}, H\'ebert,~F.,  Zhang,~H.,
  (2019)
  {\it High-accuracy mass, spin, and recoil predictions of generic black-hole merger remnants},
  \href{https://doi.org/10.1103/PhysRevLett.122.011101}{Phys.~Rev.~Lett.~{\bf 122},~011101}
  \arxiv{1809.09125}.
  \citeCount{92}
\item
  Isi,~M., {\bf Stein,~L.~C.}
  (2018)
  {\it Measuring stochastic gravitational-wave energy beyond general relativity},
  \href{https://doi.org/10.1103/PhysRevD.98.104025}{Phys.~Rev.~D~{\bf 98},~104025}
  \arxiv{1807.02123}.
  \citeCount{26}
\item
  Prabhu,~K., {\bf Stein,~L.~C.}
  (2018)
  {\it Black hole scalar charge from a topological horizon integral in
    Einstein-dilaton-Gauss-Bonnet gravity},
  \href{https://doi.org/10.1103/PhysRevD.98.021503}{Phys.~Rev.~D~{\bf 98},~021503(R)}
  (Rapid Communication)
  \arxiv{1805.02668}.
  \citeCount{46}
\item
  Gerosa,~D., H\'ebert,~F., {\bf Stein,~L.~C.}
  (2018)
  {\it Black-hole kicks from numerical-relativity surrogate models},
  \href{https://doi.org/10.1103/PhysRevD.97.104049}{Phys.~Rev.~D~{\bf 97},~104049}
  \arxiv{1802.04276}.
  \citeCount{45}
\item
  Chen,~B., {\bf Stein,~L.~C.}
  (2018)
  {\it Deformation of extremal black holes from stringy interactions},
  \href{https://doi.org/10.1103/PhysRevD.97.084012}{Phys.~Rev.~D~{\bf 97},~084012}
  \arxiv{1802.02159}.
  \citeCount{17}
\item
  Chen,~B., {\bf Stein,~L.~C.}
  (2017)
  {\it Separating metric perturbations in near-horizon extremal Kerr},
  \href{https://doi.org/10.1103/PhysRevD.96.064017}{Phys.~Rev.~D~{\bf 96},~064017}
  \arxiv{1707.05319}.
  \citeCount{8}
\item
  Okounkova,~M.,
  {\bf Stein,~L.~C.},
  Scheel,~M.~A.,
  Hemberger,~D.~A.
  (2017)
  {\it Numerical binary black hole mergers in dynamical Chern-Simons:
    I. Scalar field},
  \href{https://doi.org/10.1103/PhysRevD.96.044020}{Phys.~Rev.~D~{\bf 96},~044020}
  \arxiv{1705.07924}.
  \citeCount{111}
\item
  Tso,~R., Isi,~M., Chen,~Y., {\bf Stein,~L.~C.}
  (2017)
  {\it Modeling the Dispersion and Polarization Content of
    Gravitational Waves for Tests of General Relativity},
  \href{http://dx.doi.org/10.1142/9789813148505_0052}{CPT and Lorentz Symmetry: pp.~205--208}
  \arxiv{1608.01284}.
  \citeCount{5}
\item
  McNees,~R., {\bf Stein,~L.~C.}, Yunes,~N.
  (2016)
  {\it Extremal Black Holes in Dynamical Chern-Simons Gravity},
  \href{http://dx.doi.org/10.1088/0264-9381/33/23/235013}{Class.~Quantum Grav.~{\bf 33} 235013}
  \arxiv{1512.05453}.
  \citeCount{44}
\item
  Flanagan,~\'E.~\'E., Nichols,~D.~A., {\bf Stein,~L.~C.}, Vines,~J.
  (2016)
  {\it Prescriptions for Measuring and Transporting Local Angular
    Momenta in General Relativity},
  \href{http://dx.doi.org/10.1103/PhysRevD.93.104007}{Phys.~Rev.~D~{\bf 93},~104007}
  \arxiv{1602.01847}.
  \citeCount{12}
\item
  Yagi,~K., {\bf Stein,~L.~C.}
  (2016)
  {\it Black Hole Based Tests of General Relativity},
  \href{http://dx.doi.org/10.1088/0264-9381/33/5/054001}{Class.~Quantum Grav.~{\bf 33} 054001}
  \arxiv{1602.02413}.
  \citeCount{163}
\item
  Yagi,~K., {\bf Stein,~L.~C.}, Yunes, N.
  (2016)
  {\it Challenging the Presence of Scalar Charge and Dipolar Radiation
    in Binary Pulsars},
  \href{http://dx.doi.org/doi:10.1103/PhysRevD.93.024010}{Phys.~Rev.~D~{\bf 93}~024010}
  \arxiv{1510.02152}.
  \citeCount{112}
\item
  Berti, E., (5 authors), {\bf Stein,~L.~C.}, (46 more authors)
  (2015)
  {\it Testing General Relativity with Present and Future
    Astrophysical Observations},
  \href{http://dx.doi.org/10.1088/0264-9381/32/24/243001}{Class. Quantum Grav. {\bf 32} 243001}
  \arxiv{1501.07274}.
  \citeCount{1279}
\item
  Tsang,~D., Galley,~C.~R., {\bf Stein,~L.~C.}, Turner,~A.
  (2015)
  {\it ``Slimplectic'' Integrators: Variational Integrators for General Nonconservative Systems},
  \href{http://dx.doi.org/10.1088/2041-8205/809/1/L9}{ApJ {\bf 809} L9}
  \arxiv{1506.08443}.
  \citeCount{30}
\item
  Yagi,~K., {\bf Stein,~L.~C.}, Pappas,~G., Yunes,~N., Apostolatos,~T.
  (2014)
  {\it Why I-Love-Q: Explaining why universality emerges in compact objects},
  \href{http://dx.doi.org/10.1103/PhysRevD.90.063010}{Phys.~Rev.~D~{\bf 90}~063010}
  \arxiv{1406.7587}.
  \citeCount{72}
\item
  {\bf Stein,~L.~C.}
  (2014)
  {\it Rapidly rotating black holes in dynamical Chern-Simons gravity:
    Decoupling limit solutions and breakdown},
  \href{http://dx.doi.org/10.1103/PhysRevD.90.044061}{Phys.~Rev.~D~{\bf 90}~044061}
  \arxiv{1407.2350}.
  \citeCount{43}
\item
  {\bf Stein,~L.~C.}, Yagi,~K., Yunes,~N.
  (2014)
  {\it Three-Hair Newtonian Relations for Rotating Stars},
  \href{http://dx.doi.org/10.1088/0004-637X/788/1/15}{ApJ~{\bf 788}~15}
  \arxiv{1312.4532}.
  \citeCount{59}
\item
  {\bf Stein,~L.~C.}, Yagi,~K.
  (2014)
  {\it Parameterizing and constraining scalar corrections to general relativity},
  \href{http://dx.doi.org/10.1103/PhysRevD.89.044026}{Phys.~Rev.~D~{\bf 89}~044026}
  \arxiv{1310.6743}.
  \citeCount{29}
\item
  Yagi,~K., {\bf Stein,~L.~C.}, Yunes,~N., Tanaka,~T.
  (2013)
  {\it Isolated and Binary Neutron Stars in Dynamical Chern-Simons Gravity},
  \href{http://dx.doi.org/10.1103/PhysRevD.87.084058}{Phys.~Rev.~D~{\bf 87}~084058}
  \arxiv{1302.1918}.
  \citeCount{89}
\item
  Yagi,~K., {\bf Stein,~L.~C.}, Yunes,~N., Tanaka,~T.
  (2012),
  {\it Post-Newtonian, Quasi-Circular Binary Inspirals in Quadratic Modified Gravity},
  \href{http://dx.doi.org/10.1103/PhysRevD.85.064022}{Phys.~Rev.~D~{\bf 85}~064022}
  \arxiv{1110.5950}.
  \citeCount{175}
\item
  Vigeland,~S., Yunes,~N., {\bf Stein,~L.~C.}
  (2011),
  {\it Bumpy black holes in alternative theories of gravity},
  \href{http://dx.doi.org/10.1103/PhysRevD.83.104027}{Phys.~Rev.~D~{\bf 83}~104027}
  \arxiv{1102.3706}.
  \citeCount{141}
\item
  Yunes,~N., {\bf Stein,~L.~C.}
  (2011),
  {\it Nonspinning black holes in alternative theories of gravity},
  \href{http://dx.doi.org/10.1103/PhysRevD.83.104002}{Phys.~Rev.~D~{\bf 83}~104002}
  \arxiv{1101.2921}.
  \citeCount{202}
\item
  {\bf Stein,~L. C.}, Yunes,~N.
  (2011),
  {\it Effective gravitational wave stress-energy tensor in
    alternative theories of gravity},
  \href{http://dx.doi.org/10.1103/PhysRevD.83.064038}{Phys.~Rev.~D~{\bf 83}~064038}
  \arxiv{1012.3144}.
  \citeCount{62}
\item
  Lutomirski,~A., Tegmark,~M., Sanchez,~N.~J., {\bf
    Stein,~L.~C.}, Urry,~W.~L., Zaldarriaga,~M.
  (2011),
  {\it Solving the corner-turning problem for large interferometers},
  \href{http://dx.doi.org/10.1111/j.1365-2966.2010.17587.x}{MNRAS~{\bf 410}~2075}
  \arxiv{0910.1351}.
  \citeCount{9}
\item
  Sutton,~P., Jones,~G., Chatterji,~S., Kalmus,~P., Leonor,~I.,
  Poprocki,~S., Rollins,~J., Searle,~A., {\bf Stein,~L.}, Tinto,~M.,
  Was,~M.
  (2010),
  {\it X-Pipeline: an analysis package for autonomous
    gravitational-wave burst searches},
  \href{http://dx.doi.org/10.1088/1367-2630/12/5/053034}{New~J.~Phys.~{\bf 12}~053034}
  \arxiv{0908.3665}.
  \citeCount{130}
\item
  Chatterji,~S., Lazzarini,~A., {\bf Stein,~L.}, Sutton,~P.,
  Searle,~A.
  (2006),
  {\it Coherent network analysis technique for
    discriminating gravitational-wave bursts from instrumental noise},\\
  \href{http://dx.doi.org/10.1103/PhysRevD.74.082005}{Phys.~Rev.~D~{\bf 74}~082005}
  \arxiv{gr-qc/0605002}.
  \citeCount{113}
  \setcounter{pubCounter}{\value{enumi}}
\end{etaremune}

\section{\sc Unrefereed Publications}
\addtocounter{pubCounter}{-1}
\begin{etaremune}[start=\value{pubCounter}]
\item
  Galley,~C.~R., Tsang,~D., {\bf Stein,~L.~C.}
  (2014)
  {\it The principle of stationary nonconservative action for
    classical mechanics and field theories},
  \arxiv{1412.3082}.
  \citeCount{39}
\item
  {\bf Stein,~L.~C.}
  (2014),
  {\it Note on Legendre decomposition of the Pontryagin density in Kerr},
  \arxiv{1407.0744}.
  \citeCount{7}
\item
  {\bf Stein,~L.~C.}
  (2012),
  {\it Probes of Strong-field Gravity}, Ph.D. thesis at Massachusetts
  Institute of Technology
  [\href{http://hdl.handle.net/1721.1/77256}{hdl:1721.1/77256}].
\item
  Betancourt,~M., {\bf Stein,~L.~C.}
  (2011)
  {\it The Geometry of Hamiltonian Monte Carlo},
  \arxiv{1112.4118}.
  \citeCount{14}
\item
  {\bf Stein,~L.~C.}
  (2009),
  {\it Binary Inspiral Gravitational Waves from a Post-Newtonian Expansion},
  Contribution to the Wolfram Demonstrations Project,
  \url{http://demonstrations.wolfram.com/BinaryInspiralGravitationalWavesFromAPostNewtonianExpansion/}
\item
  {\bf Stein,~L.~C.}
  (2006),
  {\it Gravitational Wave Burst Source Localization in a Coherent
    Network Analysis},
  Senior thesis at California Institute of Technology
\end{etaremune}

%%%%%%%%%%%%%%%%%%%%%%%%%%%%%%%%%%%%%%%%%%%%

% Local Variables:
% mode: latex
% TeX-master: "LeoCStein.tex"
% End:

\else
%
\fi

%\newcommand{\playsymbol}{\framebox[1.3\width]{$\blacktriangleright$}}
\newcommand{\playsymbol}{$\blacktriangleright$}
\section{Invited Talks}
\secstartswithlist{}%
\begin{etaremune}
\item
  URI physics department colloquium
  \hfill{}
  November 2024
\item
  UNC physics department colloquium
  \hfill{}
  February 2024
\item
  UIUC astrophysics seminar
  \hfill{}
  December 2023
\item
  Harvard CMSA GR seminar
  \hfill{}
  October 2023
\item
  UMass Amherst, Amherst Center for Fundamental Interactions seminar
  \hfill{}
  September 2023
\item
  Albert Einstein Institute,
  ``Connecting the Dots'' panel discussion
  \hfill{}
  June 2023
\item
  Queen Mary Univ. of London,
  Gravitational memory workshop
  \hfill{}
  June 2023
\item
  Utah State University, Theoretical Physics Talks,
  \hfill{}
  March 2023
\item
  Iowa State, Physics and astronomy department colloquium,
  \hfill{}
  October 2022
\item
  UT Austin, Weinberg Institute seminar,
  \hfill{}
  October 2022
\item
  Vanderbilt, Physics and astronomy department colloquium,
  \hfill{}
  September 2022
\item
  ICERM, Advances in CS Classical and Quantum Gravity,
  \hfill{}
  May 2022
\item
  Flatiron CCA, Ringdown workshop, invited overview talk,
  \hfill{}
  February 2022
\item
  DAMTP (University of Cambridge), HEP/GR colloquium,
  \hfill{}
  January 2022
\item
  SISSA, Current challenges in gravitational physics workshop,
  \hfill{}
  April 2021
\item
  Flatiron CCA, Gravitational wave astronomy group seminar,
  \hfill{}
  January 2021
\item
  University of Birmingham, astrophysics seminar
  \hfill{}
  September 2020
\item
  Albert Einstein Institute, ACR division seminar
  \hfill{}
  July 2020
\item
  Black Hole Perturbation Toolkit, Spring 2020 workshop
  \hfill{}
  May 2020
\item
  American Physical Society Meeting
  \hfill{}
  April 2020
\item
  UVA, physics department colloquium
  \hfill{}
  November 2019
\item
  UT Dallas, physics department colloquium
  \hfill{}
  October 2019
\item
  Northwestern University, CIERA astrophysics seminar
  \hfill{}
  May 2019
\item
  ETH-ITS Zurich, ``New horizons for gravity'' workshop
  \hfill{}
  May 2018
\item
  UC San Diego, astrophysics seminar
  \hfill{}
  March 2018
\item
  UC Berkeley, 4D particle physics seminar
  \hfill{}
  March 2018
\item
  Kyoto University, YKIS2018a Symposium
  \hfill{}
  February 2018
\item
  Oakland University physics seminar
  \hfill{}
  February 2018
\item
  University of Wisconsin-Milwaukee gravity seminar
  \hfill{}
  January 2018
\item
  Caltech/JPL Gravitational-Wave (CaJAGWR) seminar
  \hfill{}
  January 2018
\item
  ICN UNAM,
  Relativity seminar
  \hfill{}
  December~2017
\item
  University of Mississippi,
  Astrophysics seminar
  \hfill{}
  November~2017
\item
  University of Florida,
  Astrophysics seminar
  \hfill{}
  November~2017
\item
  University of Nottingham,
  Mathematical Physics seminar
  \hfill{}
  July~2017
\item
  Sapienza University of Rome,
  New Frontiers in Gravitational-Wave Astrophysics
  \hfill{}
  June~2017
\item
  Rochester Institute of Technology,
  CCRG seminar
  \hfill{}
  March~2017
\item
  Penn State,
  IGC seminar
  \hfill{}
  March~2017
\item
  University of Mississippi,
  Strong Gravity/Binary Dynamics workshop
  \hfill{}
  February/March~2017
\item
  SUNY Stony Brook,
  ``The universe through gravitational waves''
  \hfill{}
  December~2016
\item
  University of Pennsylvania,
  New Frontiers in Gravitational Radiation workshop
  \hfill{}
  December~2016
\item
  Cambridge MA,
  Event Horizon Telescope collaboration meeting
  \hfill{}
  November/December~2016
\item
  Northwestern University CIERA,
  ``Fellows at the Frontiers''
  \hfill{}
  August/September~2016
\item
  Princeton University,
  GR@100++ panel discussion
  \hfill{}
  April 2016
\item
  Cambridge MA,
  Einstein fellows symposium
  \hfill{}
  October 2014
\item
  Perimeter Institute,
  Strong gravity seminar
  \hfill{}
  October 2014
\item
  Cornell University,
  Friends of astronomy outreach event
  \hfill{}
  November 2013
\item
  Cambridge MA,
  Einstein fellows symposium
  \hfill{}
  October 2013
\item
  SUNY Geneseo,
  Physics colloquium
  \hfill{}
  October 2013
\item
  University of Maryland,
  UMD gravity seminar
  \hfill{}
  October 2013
\item
  Yale University,
  YCAA seminar
  \hfill{}
  September 2013
\item
  Kyoto University,
  YITP long-term workshop
  \hfill{}
  June 2013
\item
  Cambridge MA,
  Einstein fellows symposium
  \hfill{}
  October 2012
\item
  Cornell University,
  Relativity lunch
  \hfill{}
  November 2011
\end{etaremune}

\section{Contributed Talks (selected)}
\secstartswithlist{}%
\begin{etaremune}
\item
  28$^{th}$ Capra Meeting on Radiation Reaction in General Relativity
  \hfill{}
  July 2025
\item
  24$^{th}$ International meeting on GR (GR24)
  \hfill{}
  July 2025
\item
  American Physical Society Meeting
  \hfill{}
  April 2024
\item
  American Physical Society Meeting
  \hfill{}
  April 2023
\item
  LISA Symposium XIV
  \hfill{}
  July 2022
\item
  American Physical Society Meeting
  \hfill{}
  April 2021
\item
  American Physical Society Meeting
  \hfill{}
  April 2019
\item
  American Physical Society Meeting
  \hfill{}
  April 2018
\item
  Pacific Coast Gravity Meeting
  \hfill{}
  March 2017
\item
  American Physical Society Meeting
  \hfill{}
  \sout{April} January 2017
\item
  Testing Gravity 2017
  \hfill{}
  January 2017
\item
  21$^{st}$ International meeting on GR (GR21)
  \hfill{}
  July 2016
\item
  American Physical Society Meeting
  \hfill{}
  April 2016
\item
  Eastern Gravity Meeting
  \hfill{}
  May 2015
\item
  American Physical Society Meeting
  \hfill{}
  April 2015
\item
  NEB 16 Recent developments in gravity
  \hfill{}
  September 2014
\item
  American Physical Society Meeting
  \hfill{}
  April 2014
\item
  XXVII Texas symposium on relativistic astrophysics
  \hfill{}
  December 2013
\item
  20$^{th}$ International meeting on GR (GR20)
  \hfill{}
  July 2013
\item
  Eastern Gravity Meeting
  \hfill{}
  June 2013
\item
  American Physical Society Meeting
  \hfill{}
  April 2013
\item
  Caltech TAPIR Seminar
  \hfill{}
  December 2011
\item
  Eastern Gravity Meeting
  \hfill{}
  June 2011
\item
  American Physical Society Meeting
  \hfill{}
  April 2011
\item
  American Physical Society Meeting
  \hfill{}
  April 2010
\end{etaremune}

%%%%%%%%%%%%%%%%%%%%%%%%%%%%%%%%%%%%%%%%%%%%

% Local Variables:
% mode: latex
% TeX-master: "LeoCStein.tex"
% End:


%\newpage{}

\section{References}
\vspace*{.05in}
\parbox{\textwidth}{%
{\bf Scott~A.~Hughes,} Professor of Physics, Massachusetts Institute of Technology \\
77 Massachusetts Avenue, Bldg.\ 37-602A \\
Cambridge, MA 02139 \\
email: \href{mailto:sahughes@mit.edu}{sahughes@mit.edu} \\
office phone: \href{tel:1-617-258-8523}{1-617-258-8523}}
\par
\parbox{\textwidth}{%
{\bf Nico~Yunes,} Professor of Physics, University of Illinois\\
249 Loomis Laboratory\\
1110 West Green Street\\
Urbana, IL 61801-3003\\
email: \href{mailto:nyunes@illinois.edu}{nyunes@illinois.edu} \\
office phone: \href{tel:1-814-883-2069}{1-814-883-2069}}
\par
\parbox{\textwidth}{%
{\bf {\'E}anna~{\'E}.~Flanagan,} Professor of Physics and Astronomy,
Cornell University\\
463 Physical Sciences Building\\
Ithaca, NY 14853\\
email: \href{mailto:eef3@cornell.edu}{eef3@cornell.edu}\\
office phone: \href{tel:1-607-255-6534}{1-607-255-6534}}
\par
\parbox{\textwidth}{%
{\bf Yanbei~Chen,} Professor of Physics,
California Institute of Technology\\
TAPIR 350-17, Caltech\\
1200 E.\ California Boulevard\\
Pasadena, CA 91125\\
email: \href{mailto:yanbei@caltech.edu}{yanbei@caltech.edu}
(please send correspondence to \href{mailto:joann@caltech.edu}{joann@caltech.edu})
\\
office phone: \href{tel:1-626-395-4258}{1-626-395-4258}}

\end{resume}
\end{document}

% Local Variables:
% mode: latex
% End:
